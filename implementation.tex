%%%%%%%%%%%%%%%%%%%%%%%%%%%%%%%%%%%%%%%%%%%%%%%%%%%%%%%%%%%%%%%%%
\chapter{Implementation}\label{ch:impl}
%%%%%%%%%%%%%%%%%%%%%%%%%%%%%%%%%%%%%%%%%%%%%%%%%%%%%%%%%%%%%%%%%
\section{Frame Works}
Deep Neural Networks needs high computations and loads of linear algebra processes. So for implementing them we need to have highly sophisticated libraries that can work with both CPU and GPU to get the best result out of it. There are plenty of libraries designed for this purpose; most famous tools are: Caffe\cite{caffe}, TensorFlow\cite{tensorflow}, Theano\cite{theano} and Torch\cite{torch}.

\subsection{Caffe}
Caffe is written in C++ and have python and matlab wrappers.
\subsection{Tensor Flow}
Tensorflow is develeped by Google and has a python interface but most of the functionalities are implemented in C++.
\subsection{Theano}
Theano is a symbolic platform for deep neural networks in python.
\subsection{Torch}
Torch is been developed in Lua and mainly used in facebook and Google DeepMind. (recently deepmind migrated to tensorflow.)